%%%%%%%%%%%%%%%%%%%%%%%%%%%%%%%%%%%%%%%%%%%%%%%%%%%%%%%%%%%%%%%
%
% Welcome to Overleaf --- just edit your LaTeX on the left,
% and we'll compile it for you on the right. If you open the
% 'Share' menu, you can invite other users to edit at the same
% time. See www.overleaf.com/learn for more info. Enjoy!
%
%%%%%%%%%%%%%%%%%%%%%%%%%%%%%%%%%%%%%%%%%%%%%%%%%%%%%%%%%%%%%%%
\documentclass[11pt, nopagenumbers]{adamblan-hw}

% Some frequently helpful commands. Feel free to add more of your own.
\newcommand{\NN}{\mathbb{N}}
\newcommand{\ZZ}{\mathbb{Z}}
\newcommand{\QQ}{\mathbb{Q}}
\newcommand{\RR}{\mathbb{R}}

\newcommand{\limit}{\lim\limits}
\newcommand{\limn}{\limit_{n \to \infty}}

% usage \set{1, 2, 3} for {1, 2, 3}. Will automatically resize the braces too.
\newcommand{\set}[1]{\left\{ #1 \right\}}
% usage \abs{\frac{1}{2}} for |1/2|. Will automatically resize the bars too.
\newcommand{\abs}[1]{\left| #1 \right|}
% same as \abs but with double bars like for vector magnitude
\newcommand{\norm}[1]{\left\| #1 \right\|}
% usage \ceil{\frac{1}{2}} for \lceil 1/2 \rceil. Will automatically resize the bars too.
\newcommand{\ceil}[1]{\left\lceil #1 \right\rceil}
% usage \floor{\frac{1}{2}} for \lfloor 1/2 \rfloor. Will automatically resize the bars too.
\newcommand{\floor}[1]{\left\lfloor #1 \right\rfloor}
% usage \round{\frac{1}{2}} for \lfloor 1/2 \rceil. Will automatically resize the bars too.
\newcommand{\round}[1]{\left\lfloor #1 \right\rceil}
% The following commands are defined via the xparse package, which you can look up for more info if you want.

% \par(\frac{1}{2}) will automatically resize the parenthesis
\NewDocumentCommand{\paren}{r()}{\ensuremath{\left( #1 \right)}}
% \brac[\frac{1}{2}] will automatically resize the brackets
\NewDocumentCommand{\brac}{r[]}{\ensuremath{\left[ #1 \right]}}

%%%%%%%%%%%%%%%%% Identifying Information %%%%%%%%%%%%%%%%%
%% DO NOT INCLUDE YOUR NAME ANYWHERE IN THE PDF. WE WANT %%
%% TO GRADE ANONYMOUSLY TO AVOID BIAS!!!!                %%
%%%%%%%%%%%%%%%%%%%%%%%%%%%%%%%%%%%%%%%%%%%%%%%%%%%%%%%%%%%

\begin{document}

\begin{question}{Inclusion-Exclusion}
How many ways are there to pick three numbers from three sets A, B, C such that:
\begin{itemize}
    \item A = $\{1, 2, 3, \dots, n\}$ 
    \item B = $\{1, 2, 3, \dots, m\}$
    \item C = $\{1, 2, 3, \dots, l\}$
    \item At least one of the numbers is a one
\end{itemize}
We use the inclusion Exclusion principle and calculate this by first looking at 
the total number of ways there are to pick three numbers from the three sets and
then subtracting from that the number of ways to pick three numbers from the three 
sets where none of them is a one.

To find the total number of ways to pick three numbers from the three sets without any restriction, we note
that there are $n$, $m$, and $l$ ways to pick from $A, B, C$ respectively. By the rule of product, 
we can say that there are $n * m * l$ ways to pick three numbers from the three sets without any restriction.

To calculate the total number of ways to pick three numbers from three sets where none of the numbers are $1$, 
we note that we're essentially excluding $1$ from each set. As such, there are $n - 1, m - 1$, and $l - 1$ ways
to pick from $A, B, C$ respectively. By the rule of product, we can say that there are $(n - 1)(m - 1)(l - 1)$ ways
to pick three numbers from the three sets where none of the numbers are $1$.

Now, to find the number of ways three numbers from the three sets where at least one of them is a one,
we subtract the numbers of ways there are to pick no ones from the total number of ways to pick three numbers.
As such, the total number of ways to pick three numbers from the three sets where at least one of them is a one
is \framebox{\text{$nml - (n - 1)(m - 1)(l - 1)$}}.
\end{question}

\end{document}