%%%%%%%%%%%%%%%%%%%%%%%%%%%%%%%%%%%%%%%%%%%%%%%%%%%%%%%%%%%%%%%
%
% Welcome to Overleaf --- just edit your LaTeX on the left,
% and we'll compile it for you on the right. If you open the
% 'Share' menu, you can invite other users to edit at the same
% time. See www.overleaf.com/learn for more info. Enjoy!
%
%%%%%%%%%%%%%%%%%%%%%%%%%%%%%%%%%%%%%%%%%%%%%%%%%%%%%%%%%%%%%%%
\documentclass[11pt, nopagenumbers]{adamblan-hw}

\usepackage{xcolor}

% Some frequently helpful commands. Feel free to add more of your own.
\newcommand{\NN}{\mathbb{N}}
\newcommand{\ZZ}{\mathbb{Z}}
\newcommand{\QQ}{\mathbb{Q}}
\newcommand{\RR}{\mathbb{R}}

\newcommand{\limit}{\lim\limits}
\newcommand{\limn}{\limit_{n \to \infty}}

% usage \set{1, 2, 3} for {1, 2, 3}. Will automatically resize the braces too.
\newcommand{\set}[1]{\left\{ #1 \right\}}
% usage \abs{\frac{1}{2}} for |1/2|. Will automatically resize the bars too.
\newcommand{\abs}[1]{\left| #1 \right|}
% same as \abs but with double bars like for vector magnitude
\newcommand{\norm}[1]{\left\| #1 \right\|}
% usage \ceil{\frac{1}{2}} for \lceil 1/2 \rceil. Will automatically resize the bars too.
\newcommand{\ceil}[1]{\left\lceil #1 \right\rceil}
% usage \floor{\frac{1}{2}} for \lfloor 1/2 \rfloor. Will automatically resize the bars too.
\newcommand{\floor}[1]{\left\lfloor #1 \right\rfloor}
% usage \round{\frac{1}{2}} for \lfloor 1/2 \rceil. Will automatically resize the bars too.
\newcommand{\round}[1]{\left\lfloor #1 \right\rceil}
% The following commands are defined via the xparse package, which you can look up for more info if you want.

% \par(\frac{1}{2}) will automatically resize the parenthesis
\NewDocumentCommand{\paren}{r()}{\ensuremath{\left( #1 \right)}}
% \brac[\frac{1}{2}] will automatically resize the brackets
\NewDocumentCommand{\brac}{r[]}{\ensuremath{\left[ #1 \right]}}

%%%%%%%%%%%%%%%%% Identifying Information %%%%%%%%%%%%%%%%%
%% DO NOT INCLUDE YOUR NAME ANYWHERE IN THE PDF. WE WANT %%
%% TO GRADE ANONYMOUSLY TO AVOID BIAS!!!!                %%
%%%%%%%%%%%%%%%%%%%%%%%%%%%%%%%%%%%%%%%%%%%%%%%%%%%%%%%%%%%

\begin{document}

\begin{question}{Rolling Dice}
For both parts, we use the definition that two events $A$ and $B$ are independent if and only if
\\Pr($A | B$) $=$ Pr($A$) and Pr($B | A$) $=$ Pr($B$). 
\begin{part} By the construction of \texttt{\textcolor{magenta}{RollDie}(6)}, we can say that 
Pr($D_1) = $Pr($D_2) = \frac{1}{6}$. By the law of product, Pr($D_1 \cap D_2) = \frac{1}{6} * \frac{1}{6} = \frac{1}{36}$.
We now perform the following calculations.

\begin{align*}
\text{Pr}(D_1 | D_2) &= \frac{\text{Pr}(D_1 \cap D_2)}{\text{Pr}(D_2)} && \text{[definition of Pr($A | B$)]}\\
&=\frac{1/36}{1/6} \\
&= \frac{1}{6} \\
&= \text{Pr}(D_2) \\
\end{align*}

Using the commutativity $\cap$, a similar calculation is performed to show that \text{Pr}($D_2 | D_1$) = Pr($D_1$). Therefore, by the definition of independence, \fbox{$D_1$ and $D_2$ are independent.}
\end{part} 

% Independence of D_1 and S_5
\begin{part} There are 36 possible outcomes for \texttt{die1 + die2} (by law of product), and only 4 of those outcomes
result in $5$, i.e. $|S_5| = |\{(1, 4), (2, 3), (3, 2), (4, 1)\}| = 4$, so Pr($S_5) = \frac{4}{36} = \frac{1}{9}$. By looking at
all the possibilities of $S_5$, we see that there is only one outcome where \texttt{die1} $= 1$, so Pr($D_1 \cap S_5) = \frac{1}{36}$.

\begin{align*}
\text{Pr}(D_1 | S_5) &= \frac{\text{Pr}(D_1 \cap S_5)}{\text{Pr}(S_5)} && \text{[definition of Pr($A | B$)]}\\
&=\frac{1/36}{1/9} \\
&= \frac{1}{4} \\
&\neq \text{Pr}(S_5) \\
\end{align*}

Since Pr($D_1 |S_5) \neq $ Pr($S_5)$, by the definition of independence, \\\fbox{$D_1$ and $S_5$ are not independent and therefore dependent.}
\end{part}
\end{question}

\end{document}