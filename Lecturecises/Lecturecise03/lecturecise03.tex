%%%%%%%%%%%%%%%%%%%%%%%%%%%%%%%%%%%%%%%%%%%%%%%%%%%%%%%%%%%%%%%
%
% Welcome to Overleaf --- just edit your LaTeX on the left,
% and we'll compile it for you on the right. If you open the
% 'Share' menu, you can invite other users to edit at the same
% time. See www.overleaf.com/learn for more info. Enjoy!
%
%%%%%%%%%%%%%%%%%%%%%%%%%%%%%%%%%%%%%%%%%%%%%%%%%%%%%%%%%%%%%%%
\documentclass[11pt, nopagenumbers]{adamblan-hw}

% Some frequently helpful commands. Feel free to add more of your own.
\newcommand{\NN}{\mathbb{N}}
\newcommand{\ZZ}{\mathbb{Z}}
\newcommand{\QQ}{\mathbb{Q}}
\newcommand{\RR}{\mathbb{R}}

\newcommand{\limit}{\lim\limits}
\newcommand{\limn}{\limit_{n \to \infty}}

% usage \set{1, 2, 3} for {1, 2, 3}. Will automatically resize the braces too.
\newcommand{\set}[1]{\left\{ #1 \right\}}
% usage \abs{\frac{1}{2}} for |1/2|. Will automatically resize the bars too.
\newcommand{\abs}[1]{\left| #1 \right|}
% same as \abs but with double bars like for vector magnitude
\newcommand{\norm}[1]{\left\| #1 \right\|}
% usage \ceil{\frac{1}{2}} for \lceil 1/2 \rceil. Will automatically resize the bars too.
\newcommand{\ceil}[1]{\left\lceil #1 \right\rceil}
% usage \floor{\frac{1}{2}} for \lfloor 1/2 \rfloor. Will automatically resize the bars too.
\newcommand{\floor}[1]{\left\lfloor #1 \right\rfloor}
% usage \round{\frac{1}{2}} for \lfloor 1/2 \rceil. Will automatically resize the bars too.
\newcommand{\round}[1]{\left\lfloor #1 \right\rceil}
% The following commands are defined via the xparse package, which you can look up for more info if you want.
\newcommand*{\T}[1]{\texttt{#1}}

% \par(\frac{1}{2}) will automatically resize the parenthesis
\NewDocumentCommand{\paren}{r()}{\ensuremath{\left( #1 \right)}}
% \brac[\frac{1}{2}] will automatically resize the brackets
\NewDocumentCommand{\brac}{r[]}{\ensuremath{\left[ #1 \right]}}

%%%%%%%%%%%%%%%%% Identifying Information %%%%%%%%%%%%%%%%%
%% DO NOT INCLUDE YOUR NAME ANYWHERE IN THE PDF. WE WANT %%
%% TO GRADE ANONYMOUSLY TO AVOID BIAS!!!!                %%
%%%%%%%%%%%%%%%%%%%%%%%%%%%%%%%%%%%%%%%%%%%%%%%%%%%%%%%%%%%

\begin{document}

\begin{question}{Functions on Trees}
\textbf{For all T $\in$ \texttt{Tree}, \texttt{size(T)} $\leq 2^{\texttt{height(T)}} - 1$}

\underline{Lemma:} $2^a + 2^b \leq 2 * 2^{max(a, b)}$
\begin{proof}
We prove this lemma by cases:

\underline{$a = b$:}
\vspace{-0.90cm}
\begin{align*}
2^a + 2^b &\stackrel{?}{\leq} 2 * 2^{max(a, b)} \\
2^a + 2^a &\stackrel{?}{\leq} 2 * 2^{max(a, a)} && \text{[this is true because of our assumption of $a = b$]}\\
2 * 2^a &\stackrel{?}{\leq} 2 * 2^{a} \\
1 &\leq 1 \\
\end{align*}
\vspace{-0.90cm}
\underline{$a < b$:}
\begin{align*}
2^a + 2^b &\stackrel{?}{\leq} 2 * 2^{max(a, b)} \\
2^a + 2^b &\stackrel{?}{\leq} 2 * 2^{b} \\
2^{a - b} + 1 &\stackrel{?}{\leq} 2 \\
\frac{2^a}{2^b} &\stackrel{?}{\leq} 1 \\
2^a &\leq 2^b && \text{[this is true because of our assumption of $a < b$]}\\
\end{align*}
A similar argument is made for $a > b$, thus our lemma is true.
\end{proof}
We now prove the original statement by structural induction. Let P(\T{T}) be the original claim.

\underline{Base Case:} \T{T = Nil}: \T{size(Nil)} $=0 \leq 0 = 2^0 - 1 = 2^{\T{height(T)}} - 1$

\underline{Inductive Hypothesis:} For some \texttt{L, R} $\in$ \T{T}, suppose P(\T{L}) and P(\T{R})

\underline{Induction Step:} We want to show P(\T{T}) where \T{T} = \T{Tree($x$, L, R)}.
\begin{align*}
\T{size(T)} &= \T{1 + size(L) + size(R)} && \text{[definition of \T{size(T)}]} \\
&\leq 1 + (2^{\T{height(L)}} - 1) + (2^{\T{height(R)}} - 1) && \text{[by I.H.]} \\
&= 2^{\T{height(L)}} + 2^{\T{height(R)}} - 1 \\
&\leq 2 * 2^{max(\T{height(L)}, \T{height(R)})} - 1 && \text{[by the lemma]}\\
&= 2^{1 + max(\T{height(L)}, \T{height(R)})} - 1 \\
&= 2^{\T{height(T)}} - 1 && \text{[definition of \T{height(T)}]} \\
\T{size(T)} &\leq 2^{\T{height(T)}} - 1 \\
\end{align*}
Thus we have proven the original claim by structual induction. \qed
\end{question}

\end{document}