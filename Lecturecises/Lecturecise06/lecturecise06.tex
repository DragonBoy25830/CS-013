%%%%%%%%%%%%%%%%%%%%%%%%%%%%%%%%%%%%%%%%%%%%%%%%%%%%%%%%%%%%%%%
%
% Welcome to Overleaf --- just edit your LaTeX on the left,
% and we'll compile it for you on the right. If you open the
% 'Share' menu, you can invite other users to edit at the same
% time. See www.overleaf.com/learn for more info. Enjoy!
%
%%%%%%%%%%%%%%%%%%%%%%%%%%%%%%%%%%%%%%%%%%%%%%%%%%%%%%%%%%%%%%%
\documentclass[11pt, nopagenumbers]{adamblan-hw}

% Some frequently helpful commands. Feel free to add more of your own.
\newcommand{\NN}{\mathbb{N}}
\newcommand{\ZZ}{\mathbb{Z}}
\newcommand{\QQ}{\mathbb{Q}}
\newcommand{\RR}{\mathbb{R}}

\newcommand{\limit}{\lim\limits}
\newcommand{\limn}{\limit_{n \to \infty}}

% usage \set{1, 2, 3} for {1, 2, 3}. Will automatically resize the braces too.
\newcommand{\set}[1]{\left\{ #1 \right\}}
% usage \abs{\frac{1}{2}} for |1/2|. Will automatically resize the bars too.
\newcommand{\abs}[1]{\left| #1 \right|}
% same as \abs but with double bars like for vector magnitude
\newcommand{\norm}[1]{\left\| #1 \right\|}
% usage \ceil{\frac{1}{2}} for \lceil 1/2 \rceil. Will automatically resize the bars too.
\newcommand{\ceil}[1]{\left\lceil #1 \right\rceil}
% usage \floor{\frac{1}{2}} for \lfloor 1/2 \rfloor. Will automatically resize the bars too.
\newcommand{\floor}[1]{\left\lfloor #1 \right\rfloor}
% usage \round{\frac{1}{2}} for \lfloor 1/2 \rceil. Will automatically resize the bars too.
\newcommand{\round}[1]{\left\lfloor #1 \right\rceil}
% The following commands are defined via the xparse package, which you can look up for more info if you want.

% \par(\frac{1}{2}) will automatically resize the parenthesis
\NewDocumentCommand{\paren}{r()}{\ensuremath{\left( #1 \right)}}
% \brac[\frac{1}{2}] will automatically resize the brackets
\NewDocumentCommand{\brac}{r[]}{\ensuremath{\left[ #1 \right]}}

%%%%%%%%%%%%%%%%% Identifying Information %%%%%%%%%%%%%%%%%
%% DO NOT INCLUDE YOUR NAME ANYWHERE IN THE PDF. WE WANT %%
%% TO GRADE ANONYMOUSLY TO AVOID BIAS!!!!                %%
%%%%%%%%%%%%%%%%%%%%%%%%%%%%%%%%%%%%%%%%%%%%%%%%%%%%%%%%%%%

\begin{document}

\begin{question}{High Five!}
We solve this problem by applying the lineary of expectations. We let the random variable $R$ be the number of students
who get two high fives from the same TA. We express $R$ as a sum of indicator variables. Let
$R_i$ be an indiccator for the event that the $i$-th student gets two high fives from the same TA i.e.
$R_i = 0$ is the event the $i$-th student doesn't get two high fives from the same TA and $R_i = 1$ be the
event the $i$-th student does get two high fives from the same TA. The number of students who get two high fives from 
the same TA can be expressed as $$R = R_1 + R_2 + \dots + R_n$$

The indicator values are independent from one another, but when applying the linearity of expectations, we don't need to consider whether the events are independent or not.
We can also apply the linearity of expecations because $R_1, R_2, \dots, R_n$ is a sequence of $n$ arbitrary random variables.

We can now take the expected value of both sides of the equation:
\begin{align*}
R &= R_1 + R_2 + \dots + R_n \\
E[R] &= E[R_1 + R_2 + \dots + R_n] \\
E[R] &= E[R_1] + E[R_2] + \dots + E[R_n] && \text{[linearity of expectation]} \\
\end{align*}

We can calculate $E[R_i]$ as follows. There are two outcomes that are possible here. The student either 
gets two high fives from the same TA or doesn't. The former outcome only occurs when both dice roll the same number,
and there is a $1/8$ probability that both dice roll the same number. This is derived with a simple counting argument.
There are 64 total outcomes of rolling two 8-sided diced, and only 8 of those outcomes have both of those dice being the same.
Similarly, there is a $7/8$ probability that the dice rolls are different.
\begin{align*}
E[R_i] = 0 * \frac{7}{8} + 1 * \frac{1}{8} && \text{[definition of expected value and indicator variable]} \\
E[R_i] = \frac{1}{8} \\
\end{align*}

We can now go back to our original equation:
\begin{align*}
E[R] &= E[R_1] + E[R_2] + \dots + E[R_n] && \text{[linearity of expectation]} \\
E[R] &= \frac{1}{8} + \frac{1}{8} + \dots + \frac{1}{8} && \text{by calculation above}\\
E[R] &= \frac{n}{8} && \text{[there are $n$ students]}\\
\end{align*}

Thus, through the lineary of expectations, we can see that the expected number of students who will receive two high fives from the same TA is \fbox{$\frac{n}{8}$}
\end{question}

\end{document}