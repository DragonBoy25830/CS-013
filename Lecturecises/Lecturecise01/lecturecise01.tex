%%%%%%%%%%%%%%%%%%%%%%%%%%%%%%%%%%%%%%%%%%%%%%%%%%%%%%%%%%%%%%%
%
% Welcome to Overleaf --- just edit your LaTeX on the left,
% and we'll compile it for you on the right. If you open the
% 'Share' menu, you can invite other users to edit at the same
% time. See www.overleaf.com/learn for more info. Enjoy!
%
%%%%%%%%%%%%%%%%%%%%%%%%%%%%%%%%%%%%%%%%%%%%%%%%%%%%%%%%%%%%%%%
\documentclass[11pt, nopagenumbers]{adamblan-hw}

%%%%%%%%%%%%%%%%% Identifying Information %%%%%%%%%%%%%%%%%
%% DO NOT INCLUDE YOUR NAME ANYWHERE IN THE PDF. WE WANT %%
%% TO GRADE ANONYMOUSLY TO AVOID BIAS!!!!                %%
%%%%%%%%%%%%%%%%%%%%%%%%%%%%%%%%%%%%%%%%%%%%%%%%%%%%%%%%%%%

\begin{document}

    \begin{question}{\color{red} Congruence and Mod Equivalence}
        \textbf{Let $a$ and $b$ be integers, and let $m$ be a positive integer. Then, $a \equiv_m b$ iff $a \mod m = b \mod m$.}

        To prove the iff statement, we will prove the implication in both directions.
        
        Suppose that $a \equiv_m b$. Thus, by definition of mod congruence 
        $$m | (a-b)$$
        By definition of $|$, there exists a $k \in \mathbb{Z}$ such that,
        $$a - b = mk$$
        $$a = mk + b$$
        $$a \mod m = (mk + b) \mod m$$ 
        The intuition for the next statement is as follows: the $k + l \mod m$ operation can be thought
        of travelling around one modular circle with $m$ notches $k + l$ number of times and returning the
        final position on the circle. Since we make a complete loop every $m$ steps, they don't contribute
        to the final position of the on the circle. As such, we can ignore the complete loops around the circle
        by taking the mod of each number and returning the sum of that.
        $$a\mod m = mk \mod m + b \mod m$$
        $$a \mod m = b \mod m \ \text{[it is obvious that $\frac{m * k}{m}$ will have a remainder of 0]}$$

        Thus, we have shown that $a \equiv_m b \Rightarrow a \mod m = b \mod m$.\\

        Now we will prove the implication in the other direction. Suppose that $a \mod m = b \mod m$.
        Using this and the definition of $\mod m$, we can say that there exists a $k_0, k_1 \in \mathbb{Z}$
        such that $a = mk_0 + a \mod m$ (1) and $b = mk_1 + b \mod m$ (2).
        \begin{align*}
        &a - b = (mk_0 + a \mod m) - (mk_1 + b \mod m)&& \text{[(1) $-$ (2)]} \\
        &a - b = m(k_0 - k_1) + (a \mod m - b \mod m) \\
        &a - b = m(k_0 - k_1) && (a \mod m = b \mod m)\\
        &m | (a - b) && \text{[definition of | since $k_0 - k_1 \in \mathbb{Z}$]}\\
        &a \equiv_m b && \text{[definition of mod congruence]}
        \end{align*}

        As such, we have shown that $a \mod m = b \mod m \Rightarrow a \equiv_m b$.

        Since we have proved the implication in both directions, we have proved the iff
        statement that $a \equiv_m b$ iff $a \mod m = b \mod m$.
        \qed
    \end{question}

    \begin{question}{\color{red} Adding Congruences}
        \textbf{Let $a, b, c, d \in \mathbb{Z}$, and let $m$ be a positive integer.
        Then, if $a \equiv_m b$ and $c \equiv_m d$, then $a + c \equiv_m b + d$.}

        We start by unrolling the definition of $\equiv_m$ and writing that
        $$m | (a - b)$$
        $$m | (c - d)$$
        By the definition of $|$, there exists a $k_0, k_1 \in \mathbb{Z}$ such that
        $$a = mk_0 + b \ (1)$$
        $$c = mk_1 + d \ (2)$$
        We can now perform the follow calculations:
        \begin{align*}
            a + c = (mk_0 + b) + (mk_1 + d) && \text{[(1) + (2)]} \\
            a + c - b - d = mk_0 + mk_1 \\
            (a + c) - (b + d) = m(k_0 - k_1) && \text{[associativity of arithmetic expressions]} \\
            m | [(a + c) - (b + d)] && \text{[definition of | since $k_0 - k_1 \in \mathbb{Z}$]} \\
            a + c \equiv_m (b + d) && \text{[definition of mod congruence]} \\
        \end{align*}
        
        Thus, we have shown that if $a \equiv_m b$, then $a + c \equiv_m b + d$.
        \qed
    \end{question}

\end{document}