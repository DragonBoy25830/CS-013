%%%%%%%%%%%%%%%%%%%%%%%%%%%%%%%%%%%%%%%%%%%%%%%%%%%%%%%%%%%%%%%
%
% Welcome to Overleaf --- just edit your LaTeX on the left,
% and we'll compile it for you on the right. If you open the
% 'Share' menu, you can invite other users to edit at the same
% time. See www.overleaf.com/learn for more info. Enjoy!
%
%%%%%%%%%%%%%%%%%%%%%%%%%%%%%%%%%%%%%%%%%%%%%%%%%%%%%%%%%%%%%%%
\documentclass[11pt, nopagenumbers]{adamblan-hw}

%%%%%%%%%%%%%%%%% Identifying Information %%%%%%%%%%%%%%%%%
%% DO NOT INCLUDE YOUR NAME ANYWHERE IN THE PDF. WE WANT %%
%% TO GRADE ANONYMOUSLY TO AVOID BIAS!!!!                %%
%%%%%%%%%%%%%%%%%%%%%%%%%%%%%%%%%%%%%%%%%%%%%%%%%%%%%%%%%%%
\begin{document}
\begin{question}{Your Average Induction}

\textbf{Let A be an arbitrary, non-empty list[int], and let n = len(A).}

\textbf{Prove that \texttt{average(A, n)} = $\sum\limits_{i=0}^{n-1}{\frac{A_i}{n}}$, 
where $A_i = A[i]$ for $0 \leq i < n$ for all $n \geq 1$.}

We start this proof by converting the code function to a mathematical function. Then, we'll proceed by induction.
\texttt{average(A, b)}is equivalent to the following function definition: $$f(n) = \frac{(n - 1) * f(n-1) + g_{n-1}}{n}$$
with initial value $f(1) = g_0$. The definition above follows from making each element in \texttt{A} an array of $g$ values 
(i.e. \texttt{A} = $[g_0, g_1, g_2, \cdots, g_n]$) and rewriting the recursion in the function as a
recursive mathematical function. The initial condition was set by the base case of \texttt{average(A, n)}. \\

Let $P(n)$ be ``$f(n) = \sum\limits_{i=0}^{n-1}{\frac{g_i}{n}}$". We prove $P(n)$ for all
$n \in \mathbb{N}$ by induction on $n$.

\underline{Base Case ($n=1$)}: $f(1) = g_0 = \frac{g_0}{1} = \sum\limits_{i=0}^{0}{\frac{g_i}{1}}$. So, $P(0)$ is true.

\underline{Inductive Hypothesis:} Suppose that $P(k)$ is true for some $k \in \mathbb{N}$.

\underline{Induction Step:} We want to show that $P(k + 1)$ is true.

\begin{align*}
f(k + 1) &= \frac{k * f(k) + g_k}{k + 1} && \text{[left side of $P(k + 1)$]} \\
&= \frac{k * \sum\limits_{i = 0}^{k-1}{\frac{g_i}{k}} + g_k}{k + 1} && \text{[by I.H.]} \\
&= \frac{k * \frac{1}{k}\sum\limits_{i = 0}^{k-1}{g_i} + g_k}{k + 1} \\
&= \frac{\sum\limits_{i = 0}^{k-1}{g_i} + g_k}{k + 1} \\
&= \frac{(g_0 + g_1 + g_2 + \cdots + g_{k-1}) + g_k}{k + 1} \\
&= \frac{1}{k + 1} \sum_{i=0}^{k}{g_i} \\
&= \sum_{i=0}^{k}{\frac{g_i}{k + 1}}
\end{align*}
So, $P(k) \Rightarrow P(k + 1)$ for all $k \in \mathbb{N}$.
It follows that $P(n)$ is true for all $n \in \mathbb{N}$ by induction.
\end{question}

\begin{question}{No, You're Being Irrational}
\textbf{Prove that $\sqrt{2} + \sqrt{5}$ is irrational. You may use the fact that $\sqrt{2}$ is irrational, but you don't have to.}

We proceed by performing a proof by contradiction. Assume that $\sqrt{2} + \sqrt{5}$ is rational.
By the definition of the rational numbers, there exists an $a, b \in \mathbb{N}, b\neq 0$ such that
$\sqrt{2} + \sqrt{5} = \frac{a}{b}$.

We perform the following calculations:
\begin{align*}
\sqrt{2} + \sqrt{5} = \frac{a}{b} && \text{[defintion of rational numbers]} \\
\sqrt{5} = \frac{a}{b} - \sqrt{2} \\
5 = \frac{a^2}{b^2} - 2\sqrt{2}\frac{a}{b} + 2 \\
2\sqrt{2}\frac{a}{b} = \frac{a^2}{b^2} - 3 \\
\sqrt{2} = \frac{a}{2b} - \frac{3b}{2a} \\
\sqrt{2} = \frac{2a^2 - 6b^2}{4ab}
\end{align*}

Since $\sqrt{2} + \sqrt{5}$ is strictly positive, we know that $a \neq 0$,
and by our use of the definition of rational numbers, we know that $b \neq 0$.
As such, $4ab \neq 0$. Since $a, b \in \mathbb{N}$, $4ab \in \mathbb{N}$ and $2a^2 - 6b^2 \in \mathbb{N}$.
Since we wrote $\sqrt{2}$ as the quotient of two integers, and the denominator is not zero,
the defintion of a rational number is satisfied implying that $\sqrt{2}$ is a rational number.
This contradicts the fact that $\sqrt{2}$ is irrational, so by contradiction, our orriginal assumption
was incorrect.

Therefore, $\sqrt{2} + \sqrt{5}$ is irrational.

\end{question}

\begin{question}{Prime Examples}
\textbf{Prove that for any prime $p > 3$, either $p \equiv_6 1$ or $p \equiv_6 5$.}

We prove the statement by proving the contrapositive which states that for any composite $p > 3$
if $p \not\equiv_6 1$ or $p \not\equiv_6 5$ . Since $\equiv_6$ can only return a
$0, 1, 2, 3, 4, \text{or a } 5$, we can investigate each case (excluding $1$ and $5$
by our claim) and show that $p$ is composite.

\begin{itemize}
\item{$p \equiv_6 0$}
\begin{align*}
6 | (p - 0) && \text{[definition of $\equiv_m$]} \\
6 | p
\end{align*}
By the definition of a factor, $6$ is a factor of $p$. Since there are factors of
$p$ other than $1$ and $p$, $p$ is composite. 

\item{$p \equiv_6 2$}
\begin{align*}
    6 | (p - 2) && \text{[definition of $\equiv_m$]} \\
    (p - 2) = 6k && \text{[for some $k \in \mathbb{Z}$ by defintion of | ]} \\
    p = 6k + 2 \\
    p = 2 * (3k + 1) \Rightarrow
    2 | p && \text{[definition of |]}
    \end{align*}
    By the definition of a factor, $2$ is a factor of $p$. Since there are factors of
    $p$ other than $1$ and $p$, $p$ is composite. 

\item{$p \equiv_6 3$}
\begin{align*}
    6 | (p - 3) && \text{[definition of $\equiv_m$]} \\
    (p - 3) = 6k && \text{[for some $k \in \mathbb{Z}$ by defintion of | ]} \\
    p = 6k + 3 \\
    p = 3 * (2k + 1) \Rightarrow
    3 | p && \text{[definition of | ]}
    \end{align*}
    By the definition of a factor, $3$ is a factor of $p$. Since there are factors of
    $p$ other than $1$ and $p$, $p$ is composite. 

\item{$p \equiv_6 4$}
    \begin{align*}
        6 | (p - 4) && \text{[definition of $\equiv_m$]} \\
        (p - 4) = 6k && \text{[for some $k \in \mathbb{Z}$ by defintion of | ]} \\
        p = 6k + 4 \\
        p = 2 * (3k + 2) \Rightarrow
        4 | p && \text{[definition of | ]}
    \end{align*}
        By the definition of a factor, $4$ is a factor of $p$. Since there are factors of
        $p$ other than $1$ and $p$, $p$ is composite. 
\end{itemize}

As such, we have proved the contrapositive and thus the original statement. 
\end{question}

\end{document}